\documentclass[11pt,largemargins]{homework}
\newcommand{\hwname}{Milan Andreew}
\newcommand{\hwemail}{milanand}
\newcommand{\hwtype}{Hausaufgabe}
\newcommand{\hwnum}{4}
\newcommand{\hwclass}{MSG:}
\newcommand{\hwlecture}{17.06.}
\newcommand{\hwsection}{2020}

\begin{document}
\maketitle

\question
\textit{Weise nach: Für jede natürliche Zahl \(n\) ist \(n^3+(n+1)^3+(n+2)^3\) durch \(9\) teilbar.}
\begin{align*}
(n+1)^3 + (n+2)^3 + (n+3)^3 &= (n+1)^3 + (n+2)^3 + \Bigl( n^3 + 9n^2 + 27n + 81\Bigr)\\
&= n^3 + (n+1)^3 + (n+2)^3 + \Bigl( 9n^2 + 27n + 81\Bigr)
\end{align*}
\textbf{Wenn} \(n^3 + (n+1)^3 + (n+2)^3\) durch 9 teilbar ist, \textbf{dann} ist \((n+1)^3 + (n+2)^3 + (n+3)^3\) auch durch 9 teilbar.
\question
\textit{Weise nach: Für jede natürliche Zahl \(n\) ist \(11^{n+2}+12^{2n+1}\) durch \(133\) teilbar.}
\begin{align*}
11^{n+2}+12^{2n+1}=11^2\cdot 11^n+12\cdot 12^{2n} = 121 \cdot 11^n + 12 \cdot 144^n \\
= 121\cdot 11^n + 12\cdot(133+11)^n \\
\Longleftrightarrow 121\cdot 11^n+12\cdot11^n = 133\cdot 11^n \\
\Longleftrightarrow 0 (\mod 133)
\end{align*}
\question
\begin{alphaparts}
	\questionpart \textit{Begründe die Schritte der folgenden Umformung:}
\end{alphaparts}
\begin{alphaparts}
	\questionpart \textit{Beweise durch vollständige Induktion unter Verwendung der in a) hergeleiteten Ungleichung, dass für alle natürlichen Zahlen \(n\geq2\) gilt: \(\frac{1}{\sqrt{1}}+\frac{1}{\sqrt{2}}+\cdots+\frac{1}{\sqrt{n}}>\sqrt{n}\)}
\end{alphaparts}
\end{document}
